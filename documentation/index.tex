%%%%%%%%%%%%%%%%%%%%%%%%%%%%%%%%%%%%%%%%%%%%%%%%%%%%%%%%%%%%%%%%%%%%%%%%%%%%%%%%%%%%%%%%%%%%%%%%%%%
%
% Primordial Machine's Math Matrices Library
% Copyright (c) 2017-2019 Michael Heilmann
%
% This software is provided 'as-is', without any express or implied warranty.
% In no event will the authors be held liable for any damages arising from the
% use of this software.
%
% Permission is granted to anyone to use this software for any purpose,
% including commercial applications, and to alter it and redistribute it
% freely, subject to the following restrictions:
%
% 1. The origin of this software must not be misrepresented;
%    you must not claim that you wrote the original software.
%    If you use this software in a product, an acknowledgment
%    in the product documentation would be appreciated but is not required.
%
% 2. Altered source versions must be plainly marked as such,
%    and must not be misrepresented as being the original software.
%
% 3. This notice may not be removed or altered from any source distribution.
%
%%%%%%%%%%%%%%%%%%%%%%%%%%%%%%%%%%%%%%%%%%%%%%%%%%%%%%%%%%%%%%%%%%%%%%%%%%%%%%%%%%%%%%%%%%%%%%%%%%%

\documentclass[oneside]{book}

\input{common}
\SetOrganization{Primordial Machine}
\SetLibraryName{Math Matrices}
\SetLibraryVersion{1.2}
\SetLibraryRepository{https://github.com/primordialmachine/math-matrices}
\SetAuthor{Michael Heilmann}
\SetEmail{michaelheilmann@primordialmachine.com}


\SetLibraryIncludeFileName{include.hpp}
\SetLibraryIncludesDirectoryPath{primordialmachine/math-matrices/\newline\$(PlatformTarget.toLower())/\$(Configuration.toLower())/includes}

\SetLibraryIncludeDirectiveFilePath{primordialmachine/math/matrices/include.hpp}

\SetLibraryStaticLibrariesDirectoryPath{primordialmachine/math-matrices/\newline\$(PlatformTarget.toLower())/\$(Configuration.toLower())/libraries}
\SetLibraryStaticLibraryFileName{math-matrices.lib}

\SetDocumentType{User Manual}

\begin{document}

\frontmatter

\begin{titlepage}
\maketitle
\end{titlepage}

\tableofcontents
\addtocontents{toc}{\protect\thispagestyle{empty}}
\pagenumbering{gobble}

\mainmatter

\chapter{Synopsis}
C++ 17 library providing $n \times m$ floating point matrices. 
The library is made available publicly on
\href{\GetLibraryRepository}{Github}
under the
\href{\GetLibraryRepository/blob/master/LICENSE}{MIT License}.

\chapter{Limitations and Restrictions}
The library officially only supports Visual Studio 2017 and Windows 10.

\chapter{Introductory example}
\textit{\color{orange}This library does not provide any examples yet.}
%Examples are located in the \href{\GetLibraryRepository/blob/master/examples}{examples} directory.

\input{building_visual_studio_2017}

\chapter{Library Interface Documentation}

\section{\texttt{namespace primordialmachine}}
The namespace this library is adding its declarations/definitions to.
The added namespace elements are documented below.

%%%%%%%%%%%%%%%%%%%%%%%%%%%%%%%%%%%%%%%%%%%%%%%%%%%%%%%%%%%%%%%%%%%%%%%%%%%%%%%%%%%%%%%%%%%%%%%%%%%
%
% Primordial Machine's Math Points Library
% Copyright (C) 2017-2019 Michael Heilmann
%
% This software is provided 'as-is', without any express or implied warranty.
% In no event will the authors be held liable for any damages arising from the
% use of this software.
%
% Permission is granted to anyone to use this software for any purpose,
% including commercial applications, and to alter it and redistribute it
% freely, subject to the following restrictions:
%
% 1. The origin of this software must not be misrepresented;
%    you must not claim that you wrote the original software.
%    If you use this software in a product, an acknowledgment
%    in the product documentation would be appreciated but is not required.
%
% 2. Altered source versions must be plainly marked as such,
%    and must not be misrepresented as being the original software.
%
% 3. This notice may not be removed or altered from any source distribution.
%
%%%%%%%%%%%%%%%%%%%%%%%%%%%%%%%%%%%%%%%%%%%%%%%%%%%%%%%%%%%%%%%%%%%%%%%%%%%%%%%%%%%%%%%%%%%%%%%%%%%

% Macro to emit the is_* and is_*_v definitions which are used by e.g. angle, color, matrix, point, and vector types.
\newcommand{\Isa}[1]{%
\section{\texttt{is\_#1} (struct template)}
Derives from \texttt{true\_type} is the specified type is a #1 type, derives from \texttt{false\_type} otherwise.%

\subsection{\texttt{is\_#1\_v} (helper variable template)}%
\texttt{template \textlangle typename TYPE\textrangle\newline%
inline constexpr bool is\_#1\_v = is\_#1\textlangle TYPE\textrangle::value;}%
}
\Isa{matrix}

%%%%%%%%%%%%%%%%%%%%%%%%%%%%%%%%%%%%%%%%%%%%%%%%%%%%%%%%%%%%%%%%%%%%%%%%%%%%%%%%%%%%%%%%%%%%%%%%%%%
\section{\texttt{matrix} (struct)}
\begin{verbatim}
template<typename TRAITS, typename ENABLED = void>
struct matrix;
\end{verbatim}
\noindent{}This library provides specializations where \texttt{TRAITS} is a \texttt{matrix\_traits}
specialization with the following properties:
\begin{enumerate}
	\item \texttt{ELEMENT\_TYPE} is a type \texttt{TYPE} for which \texttt{is\_scalar\textlangle TYPE \textrangle::value} is \texttt{true}.
	\item \texttt{NUMBER\_OF\_ROWS} and \texttt{NUMBER\_OF\_COLUMNS} each are of any number between 0 and
	\texttt{std::numeric\_limits\textlangle size\_t\textrangle\::max()}.
\end{enumerate}

\subsection{Members}

\subsection{Arithmetic Functors Support}

%%%%%%%%%%%%%%%%%%%%%%%%%%%%%%%%%%%%%%%%%%%%%%%%%%%%%%%%%%%%%%%%%%%%%%%%%%%%%%%%%%%%%%%%%%%%%%%%%%%
\subsubsection{\texttt{operator+} (binary, matrix matrix)}
This library provides specializations of
\texttt{binary\_plus\_functor}
(see \cite{arithmeticfunctors}) to compute the
sum
of two matrices. Both type arguments must be the same matrix type \texttt{M}, the result type is \texttt{M}.

\subsubsection{\texttt{operator+} (unary, matrix)}
This library provides specializations of
\texttt{unary\_plus\_functor}
(see \cite{arithmeticfunctors}) to compute
the affirmation
of a matrix.
The type argument must be a matrix type \texttt{M}, the result type is \texttt{M}.

\subsubsection{\texttt{operator+=} (compound binary, matrix matrix)}
This library provides specializations of
\texttt{plus\_assignment\_functor}
(see \cite{arithmeticfunctors}) to compute
the sum of to matrices and assign the result to the first matrix.
The type arguments must both the same matrix type \texttt{M}, the result type is \texttt{M}.

%%%%%%%%%%%%%%%%%%%%%%%%%%%%%%%%%%%%%%%%%%%%%%%%%%%%%%%%%%%%%%%%%%%%%%%%%%%%%%%%%%%%%%%%%%%%%%%%%%%
\subsubsection{\texttt{operator-} (binary, matrix matrix)}
This library provides specializations of
\texttt{binary\_minus\_functor}
(see \cite{arithmeticfunctors}) to compute the
difference
of two matrices. Both type arguments must be the same matrix type \texttt{M}, the result type is \texttt{M}.

\subsubsection{\texttt{operator-} (unary, matrix)}
This library provides specializations of
\texttt{unary\_minus\_functor}
(see \cite{arithmeticfunctors}) to compute
the negation
of a matrix.
The type argument must be a matrix type \texttt{M}, the result type is \texttt{M}.

\subsubsection{\texttt{operator-=} (compound binary, matrix matrix)}
This library provides specializations of
\texttt{minus\_assignment\_functor}
(see \cite{arithmeticfunctors}) to compute
the difference of to matrices and assign the result to the first matrix.
The type arguments must both the same matrix type \texttt{M}, the result type is \texttt{M}.

%%%%%%%%%%%%%%%%%%%%%%%%%%%%%%%%%%%%%%%%%%%%%%%%%%%%%%%%%%%%%%%%%%%%%%%%%%%%%%%%%%%%%%%%%%%%%%%%%%%
\subsubsection{\texttt{operator*} (binary, matrix scalar)}
This library provides specializations of
\texttt{binary\_star\_functor}
(see \cite{arithmeticfunctors} to compute the
product of a matrix and a scalar.
The first type argument is a matrix type \texttt{M}, the second type parameter is a scalar type \texttt{S} (see \cite{mathscalars}
such that \texttt{is\_same\_v\textlangle element\_type\_v\textlangle M\textrangle, S\textrangle} holds.
The result type is \texttt{M}.

\subsubsection{\texttt{operator*=} (compound binary, matrix scalar)}
This library provides specializations of
\texttt{star\_assignment\_functor}
(see \cite{arithmeticfunctors}) to compute the
product of a matrix and a scalar and assign the result to the matrix.
The first type argument is a matrix type \texttt{M}, the second type parameter is a scalar type \texttt{S} (see \cite{mathscalars}
such that \texttt{is\_same\_v\textlangle element\_type\_v\textlangle M\textrangle, S\textrangle} holds.
The result type is \texttt{M}.


\subsubsection{\texttt{operator*} (binary, scalar matrix)}
This library provides specializations of
\texttt{binary\_star\_functor}
(see \cite{arithmeticfunctors} to compute the
product of a scalar and a matrix.
The first type parameter is a scalar type \texttt{S} (see \cite{mathscalars}), the second type argument is a matrix type \texttt{M}.
such that \texttt{is\_same\_v\textlangle element\_type\_v\textlangle M\textrangle, S\textrangle} holds.
The result type is \texttt{M}.

\subsubsection{\texttt{operator/} (binary, matrix scalar)}
This library provides specializations of
\texttt{binary\_star\_functor}
(see \cite{arithmeticfunctors} to compute the
product of a matrix and a scalar.
The first type argument is a matrix type \texttt{M}, the second type parameter is a scalar type \texttt{S} (see \cite{mathscalars}
such that \texttt{is\_same\_v\textlangle element\_type\_v\textlangle M\textrangle, S\textrangle} holds.
The result type is \texttt{M}.

\subsubsection{\texttt{operator/=} (compound binary, matrix scalar)}
This library provides specializations of
\texttt{star\_assignment\_functor}
(see \cite{arithmeticfunctors}) to compute the
quotient of a matrix and a scalar and assign the result to the matrix.
The first type argument is a matrix type \texttt{M}, the second type parameter is a scalar type \texttt{S} (see \cite{mathscalars}
such that \texttt{is\_same\_v\textlangle element\_type\_v\textlangle M\textrangle, S\textrangle} holds.
The result type is \texttt{M}.

\subsubsection{\texttt{operator*} (binary, matrix matrix)}
This library provides specializations of
\texttt{binary\_star\_functor}
(see \cite{arithmeticfunctors}) to compute the
product
of two matrices. The first type argument is a matrix type \texttt{A} and the second type argument is a matrix type \texttt{B}
such that
\begin{enumerate}
	\item \texttt{number\_of\_columns\_v\textlangle A \textrangle} equals \texttt{number\_of\_rows\_v\textlangle B\textrangle}.
	\item \texttt{is\_same\_v\textlangle element\_type\_v\textlangle A, B\textrangle\textrangle}
\end{enumerate}
The result type is a matrix type \texttt{C} such that
\begin{enumerate}
	\item \texttt{number\_of\_rows\_v\textlangle C\textrangle := number\_of\_rows\_v\textlangle A\textrangle}
	\item \texttt{number\_of\_columns\_v\textlangle C\textrangle := number\_of\_columns\_v\textlangle B\textrangle}
	\item \texttt{element\_type\_v\textlangle C\textrangle := element\_type\_v\textlangle A\textrangle}
\end{enumerate}

%%%%%%%%%%%%%%%%%%%%%%%%%%%%%%%%%%%%%%%%%%%%%%%%%%%%%%%%%%%%%%%%%%%%%%%%%%%%%%%%%%%%%%%%%%%%%%%%%%%

\subsection{Relational Functors Support}

\subsubsection{\texttt{operator==} (compound binary, matrix matrix)}
This library provides specializations of
\texttt{equal\_to\_functor}
(see \cite{relationalfunctors}) to compute
wether two matrices of the same type are equal.

\subsubsection{\texttt{operator!=} (compound binary, matrix matrix)}
This library provides specializations of
\texttt{not\_equal\_to\_functor}
(see \cite{relationalfunctors}) to compute
wether two matrices of the same type are not equal.

%%%%%%%%%%%%%%%%%%%%%%%%%%%%%%%%%%%%%%%%%%%%%%%%%%%%%%%%%%%%%%%%%%%%%%%%%%%%%%%%%%%%%%%%%%%%%%%%%%%

\subsection{\texttt{inverse\_functor} (\textit{FunctorBase})}
A \textit{FunctorBase} with a single template parameter \texttt{M} followed by the \texttt{ENABLED} template parameter.
Specializations of this \textit{FunctorBase} compute inverses of matrices.
A possible implementation is given by
\begin{verbatim}
template<typename A, typename ENABLED = void>
struct transpose_functor;
\end{verbatim}

\subsubsection{\texttt{inverse} (function)}
This library provides a function to invoke the \texttt{inverse\_functor}.
A possible implementation is given by\newline
\texttt{template\textlangle typename A\textrangle\newline
auto\newline
inverse(const A\& a)\newline
noexcept(noexcept(inverse\_functor\textlangle A \textrangle))\newline
-> decltype(inverse\_functor\textlangle A\textrangle()(operand))\newline
\{ return inverse\_functor\textlangle A\textrangle()(a); \}}

\subsection{\texttt{inverse\_functor} (\textit{Functor})}
This library provides \textit{Functor} specializations of the \texttt{inverse\_functor} \textit{FunctorBase}
where \texttt{A} is a non degenerate square matrix type and the result type is \texttt{A}.

\subsection{\texttt{transpose\_functor} (\textit{FunctorBase})}
A \textit{FunctorBase} with a single parameter \texttt{A} followed by the \texttt{ENABLED} template parameter.
Specializations of this \textit{FunctorBase} compute transposes of matrices.
A possible implementation is given by
\begin{verbatim}
template<typename A, typename ENABLED = void>
struct transpose_functor;
\end{verbatim}

\subsubsection{\texttt{transpose} (function)}
This library provides a function to invoke the \texttt{transpose\_functor}.
A possible implementation is given by\newline
\texttt{template\textlangle typename A\textrangle\newline
auto\newline
transpose(const A\& a)\newline
noexcept(noexcept(transpose\_functor\textlangle A\textrangle()(a)))\newline
-> decltype(transpose\_functor\textlangle A\textrangle()(a))\newline
\{ return transpose\_functor\textlangle A\textrangle()(a); \}}

\subsection{\texttt{transpose\_functor} (\textit{Functor})}
This library provides \textit{Functor} specializations of the \texttt{transpose\_functor} \textit{FunctorBase}
where \texttt{A} is matrix type and the result type is \texttt{B} such that
\begin{enumerate}
	\item \texttt{element\_type\_v\textlangle B\textrangle = element\_type\_v\textlangle A\textrangle}
	\item \texttt{number\_of\_rows\_v\textlangle B \textrangle = number\_of\_columns\_v\textlangle A\textrangle}
	\item \texttt{number\_of\_columns\_v\textlangle B \textrangle = number\_of\_rows\_v\textlangle A\textrangle}
\end{enumerate}
\noindent{}Note that the number of rows and columns of the result type are swapped.

\subsection{\texttt{trace\_functor} (\textit{FunctorBase})}
An \textit{FunctorBase} with a single template parameter \texttt{A} followed by the \texttt{ENALBED} parameter.
Specializations of this \textit{FunctorBase} compute traces of matrices.
A possible implementation is given by
\begin{verbatim}
template<typename A, typename ENABLED = void>
struct trace_functor;
\end{verbatim}

\subsubsection{\texttt{trace} (function)}
This library provides a function to invoke the \texttt{trace\_functor}.
A possible implementation is given by\newline
\texttt{template\textlangle typename A\textrangle\newline
auto\newline
trace(const A\& a)\newline
noexcept(noexcept(trace\_functor\textlangle A\textrangle()(a)))
-> decltype(trace\_functor\textlangle A\textrangle()(a))\newline
\{ return trace\_functor\textlangle A\textrangle()(a); \}}

\subsection{\texttt{trace\_functor} (\textit{Functor})}
This library provides \textit{Functor} specializations of the \texttt{trace\_functor} \textit{FunctorBase}
where \texttt{A} is a non degenerate matrix type and the result type is \texttt{element\_type\_v\textlangle
A\textrangle}.


\input{bibliography}

\end{document}
